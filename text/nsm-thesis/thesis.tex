\documentclass[]{nsm-thesis}
% options:
% [germanthesis] - Thesis is written in German
% [plainunnumbered] - Don't print numbers on plain pages
% [earlydraft] - Settings for quick draft printouts
% [watermark] - Print current time/date at bottom of each page
% [phdthesis] - switch to PhD thesis style
% [twoside] - double sided
% [cutmargins] - text body fills complete page

% Author name. Separate multiple authors with commas.
\author{Dmitriy Monakhov}
\birthday{28. April 1999}
\birthplace{Moscow}

% Title of your thesis.
\title{Improving Urban Traffic Flow with Drone Supported Vehicular Networks}

\thesistype{Master's Thesis in Distributed Systems Engineering}\thesiscite{Master's Thesis~(Masterarbeit)}

\advisors{Mario Franke}
% List of advisors (without academic titles), separated by commas.

% List of referees (without academic titles), separated by commas.
\referees{Christoph Sommer, Burkhard Hensel}


% Define abbreviations used in the thesis here.
%\acrodef{WSN}{Wireless Sensor Network}
\acrodef{VANET}{Vehicular Ad Hoc Network}
%\acrodef{ROI}{Region of Interest}{short-indefinite={an}, long-plural-form={Regions of Interest}}
%\acrodef{ADAC}{German Automobile Association}{foreign={Allgemeiner Deutscher Automobilclub}}
%\acrodef{CANhashing}[CAN]{Content Addressable Network}{extra={when referring to the distributed hash table}}
%\acrodef{CANproto}[CAN]{Controller Area Network}{extra={when referring to the bus protocol}}

\begin{document}

\pagenumbering{roman}

\maketitle

\cleardoublepage


\chapter*{Abstract}
\addcontentsline{toc}{chapter}{Abstract}
\begin{otherlanguage*}{american}

\TODO{Write abstract}

about 1/2 page:
\begin{enumerate}
    \item Motivation (Why do we care?)
    \item Problem statement (What problem are we trying to solve?)
    \item Approach (How did we go about it)
    \item Results (What's the answer?)
    \item Conclusion (What are the implications of the answer?)
\end{enumerate}

The abstract is a miniature version of the thesis.
It should be treated as an entirely separate document.
Do not assume that a reader who has access to an abstract will also have access to the thesis.
Do not assume that a reader who reads the thesis has read the abstract.

\end{otherlanguage*}


\chapter*{Kurzfassung}
\addcontentsline{toc}{chapter}{Kurzfassung}
\begin{otherlanguage*}{ngerman}

\TODO{Write short version of the Abstract in German?}
Gleicher Text (sinngemäß, nicht wörtlich) in Deutsch

\end{otherlanguage*}
\acresetall

\cleardoublepage
\tableofcontents
\TODO{The table of contents should fit on one page. When in doubt, adjust the \texttt{tocdepth} counter.}

\cleardoublepage
\pagenumbering{arabic}



\chapter{Introduction}
%\chapter{Einleitung}
\label{sec:introduction}

Current technology already enables us to create semi-autonomous vehicles. For now fully-autonomous vehicles are rarely allowed to 
populate the roads, in future it will probably be possible. One perspective option for autonomous driving are cooperative autonomous vehicles: such machines
that can drive on their own and maintain situation-awareness via communicating with other vehicles on the roads.

Since our world is rapidly urbanizing, it will be desirable to enable autonomous vehicles to drive in large cities. And there is a problem here: communication between cars will be difficult in urban areas. In cities, buildings stand close to each other. Buildings are usually built of radio-transparent materials. At the same time, the buildings are quite massive. Thus, it turns out that the propagation of radio signals between low-power car transmitters becomes very difficult if the cars are not on the same street. This leads to communication outages in vehicle-to-vehicles networks operating inside urban areas.

One possible solution to this problem could be drones. Cities of the future will probably be populated by autonomous drones flying in the air above the roofs of buildings. This gives drones the advantage of making it much easier for them to maintain a line of sight between each other and with vehicles on the ground. Consequently, their radio transmissions are less likely to be blocked by buildings. In this way, drones can help vehicles spread signals bypassing buildings standing between the vehicles.

In this paper, I explore whether there is an advantage in connecting a network of drones to a network of cars in terms of assisting with urban vehicular traffic.
Since a real-life experiment will be too expensive and difficult to conduct, I will use software simulation.

\chapter{Fundamentals}
\label{sec:fundamentals}

Before the beginning of the work, I will explain the basic terms, technologies and methods that will be used later. I see the Ad-hoc networks as a good starting point.

\section{Ad hoc networks}

A good definition of ad hoc networks was given by Ram Ramanathan and Jason Redi in IEEE communications Magazine \cite{ramanathan2002brief}: An ad hoc network is a (possibly mobile) collection of communications devices (nodes) that wish to communicate, but have no fixed infrastructure available, and have no pre-determined organization of available links. The authors further make a good point, stating that individual nodes are responsible for dynamically discovering which other nodes they can directly communicate with and the nodes are required to relay packets on behalf of other nodes in order to deliver data across the network.

This feature \FIXME{Move to Broadcast storm section and add source} of ad hoc networks that nodes need to relay other nodes' messages requires additional attention, because it causes the problem of broadcast storm.

According to Toor et al \cite{toor2008}, Vehicular ad hoc network (VANET) is a special case of a mobile ad hoc network, where communication links are formed between the vehicles.

\chapter{Implementation}


\begin{itemize}
\item describe everything you yourself did (as opposed to the fundamentals chapter, which explains what you built on)
\item start with a theoretical approach
\item describe the developed system/algorithm/method from a high-level point of view
\item go ahead in presenting your developments in more detail
\item make sure to instill in your readers a high degree of confidence that your system/algorithm/method is correct -- this will likely mean conducting (and reporting on) a validation/verification step
\item recommended length: approximately one third of the thesis.
\end{itemize}



\chapter{Evaluation}


\begin{itemize}
\item measurement setup / results / evaluation / discussion
\item whatever you have done, you must comment it, compare it to other systems, evaluate it
\item usually, adequate graphs help to show the benefits of your approach
\item each result/graph must not only be described, but also discussed (What's the reason for this peak? Why have you observed this effect? What does this tell about your architecture/system/implementation?)
\item recommended length: approximately one third of the thesis.
\end{itemize}



\chapter{Conclusion}


\begin{itemize}
\item summarize again what your paper did, but now emphasize more the results, and comparisons
\item write conclusions that can be drawn from the results found and the discussion presented in the paper
\item future work (be very brief, explain what, but not much how, do not speculate about results or impact)
\item recommended length: one page.
\end{itemize}



\cleardoublepage

\listofabbreviations
\clearpage

\listoffigures
\clearpage

\listoftables
\clearpage

\printbibliography

\end{document}
